\documentclass[12pt,a4paper,oneside,draft]{report}
\usepackage{ragged2e}
\usepackage{titlesec}
\usepackage{hyperref}
\usepackage[final]{listings}
% \usepackage{indentfirst}
% use \indent\indent instead
\usepackage{soul}
\usepackage{courier}
\usepackage{xcolor}
\usepackage{varwidth}
\usepackage[T1]{fontenc}
\newcommand{\fakesection}[1]{% instead of /section
  \par\refstepcounter{section}% Increase section counter
  \sectionmark{#1}% Add section mark (header)
  \addcontentsline{toc}{section}{\protect\numberline{\thesection}#1}% Add section to ToC
  % Add more content here, if needed.
}
\overfullrule=0pt

% Defines
\def\Plus{\texttt{+}}
\def\Minus{\texttt{-}}


% Custom colors
\definecolor{codehlback}{gray}{0.9}
\definecolor{codegreen}{rgb}{0.0,0.5,0.0}
\definecolor{codegray}{rgb}{0.5,0.5,0.5}
\definecolor{codepurple}{rgb}{0.58,0,0.82}

% Macros
\sethlcolor{codehlback}
\newcommand{\cc}[1]{\texttt{#1}}

% Code folder
% \lstset{inputpath=../code}

% Code style
\lstdefinestyle{clanguage}{
    language=C,
    backgroundcolor=\color{yellow!20},   
    commentstyle=\color{codegreen},
    keywordstyle=\color{blue},
    numberstyle=\tiny\color{codegray},
    stringstyle=\color{codepurple},
    basicstyle=\small\ttfamily,
	frame=single
    breakatwhitespace=false,         
    breaklines=true,                 
    keepspaces=true,                 
    numbers=left,       
    numbersep=5pt,                  
    showspaces=false,                
    showstringspaces=false,
    showtabs=false,                  
    tabsize=4,
}

\lstdefinestyle{codeoutput}{
	basicstyle=\small\ttfamily,
	numbers=none,
	frame=single,
	columns=fullflexible,
	backgroundcolor=\color{blue!10},
}

\begin{document}

\titleformat{\chapter}[display] % Removing chapter header
{\normalfont\bfseries}{}{0pt}{\Large}

\begin{center}
	Notes for Dragon Book
\end{center}

\chapter{}
\section*{Declarative vs. Imperative}
\indent\indent Broadly speaking, computer programming languages have been divided into two categories — imperative languages and declarative languages.
Examples of imperative languages are Pascal, C, Java, etc. Examples of declarative languages are ML, pure Lisp and pure Prolog.
\par
The programming model in imperative languages is based on a statement-at-a-time paradigm where each statement has some effect on a memory store. Imperative programming is centered around the assignment statement, which allows one to change the content of cells in the memory store. Programs written in imperative languages are generally harder to write, debug, and maintain compared to those written in declarative languages. Imperative programming lays more stress on "how" a solution procedure is specified. Programs written in imperative languages are generally larger in terms of code size and run faster compared to programs written in declarative languages. Imperative languages do not have a solid mathematical basis.
\par
The programming model in declarative languages is based on stating the relationship between inputs and outputs. The actual computation procedure adopted is left to the runtime system. A declarative program can be viewed as a high level specification. Declarative programs are shorter, supposedly easier to write, debug, and maintain. Declarative programs are generally slower than imperative programs in execution speed.
There are two major programming paradigms that are declarative: functional programming (with its formal basis in mathematical functions and the lambda calculus) and logic programming (with its formal basis in first order logic).

\newpage

\section*{Static vs. Dynamic scoping}

\lstset{style=clanguage}
Consider the following pseudocode.
\lstinputlisting{code/static\_dynamic\_scoping.c}

\lstset{style=codeoutput}
\noindent Dynamic scope output:
\begin{lstlisting}
20
\end{lstlisting}

\lstset{style=codeoutput}
\noindent Static scope output:
\begin{lstlisting}
10
\end{lstlisting}

Dynamic scope resolution is also essential for polymorphic procedures, those that have two or more definitions with the same name, depending only on types of the arguments.
\\\indent A language uses \emph{static scope} or \emph{lexical scope} if it is possible to determine the scope of declaration by looking only at the program. Otherwise, the language uses \emph{dynamic scope}

\newpage

\section*{Names, Identifiers and Variables}
\indent\indent An \emph{identifier} is a string of characters, typically letters or digits, that refer to (identifies) an entity, such as a data object, a procedure, a class, or a type. All identifiers are names, but not names are identifiers. Names can also be expressions. For example, the name \cc{x.y} might denote the field \cc{y} of a structure denoted by \cc{x}. Here, \cc{x} and \cc{y} are identifiers, while \cc{x.y} is a name, but not an identifier. Composite names like \cc{x.y} are called qualified names.
\indent A \emph{variable} refers to a particular location of the store. It is common for the same identifier to be declared more than once. Each such declaration introduces a new variable. Even if each identifier declared just once, an identifier local to a recursive procedure will refer to different locations of the store at different times.

\newpage
\section*{Aliasing}
\lstset{style=clanguage}
\lstinputlisting{code/alias.c}
\lstset{style=codeoutput}
Output:
\begin{lstlisting}
q: x[1]=42	 y[1]=42
t: x[1]=377	 y[1]=377
p: x=9890	 y=1
\end{lstlisting}

\newpage
\section*{Syntax Definition}
\begin{center}
\textbf{if} ( expression ) statement \textbf {else} statement \\
$\big\Downarrow$ \\
\emph{stmt} $\rightarrow$ \textbf{if} ( \emph{expr} ) \emph{stmt} \textbf{else} \emph{stmt}
\end{center}
\indent\indent Such a rule is called a \emph{production}. In a production, lexical elements like the keyword \textbf{if} and the parentheses are called terminals. Variables like \emph{expr} and \emph{stmt} represent sequences of terminals and are called \emph{nonterminals}. \\

A \emph{context-free} grammar has four components:
\begin{enumerate}
\item \emph{Terminal symbols} (\emph{tokens}) are literal symbols that may appear in the outputs of the production rules of a formal grammar and which cannot be changed using the rules of the grammar.
\item \emph{Nonterminal symbols} (\emph{syntactic variables}) are those symbols that can be replaced.
\item A set of \emph{productions}, where each production consists of nonterminal, called \emph{head} or \emph{left side}, and a sequence of terminals and/or nonterminals, called \emph{body} or \emph{right side}. Production rules may be used to generate strings, or to parse them. Each such rule has a head, or left-hand side, which consists of the string that may be replaced, and a body, or right-hand side, which consists of a string that may replace it. Rules are often written in the form head → body; e.g., the rule a → b specifies that a can be replaced by b.
\item A designation of one of the nonterminals as the \emph{start} symbol.
\end{enumerate}

\noindent Simple calculator productions:
\begin{center}
\begin{varwidth}{\textwidth}
\item[] \emph{list} $\rightarrow$ \emph{list} \Plus\ \emph{digit}
\item[] \emph{list} $\rightarrow$ \emph{list} \Minus\ \emph{digit}
\item[] \emph{list} $\rightarrow$ \emph{digit}
\item[] \emph{digit} $\rightarrow$ 0 | 1 | 2 | 3 | 4  ... 8 | 9
\end{varwidth}
\end{center}
Which can be grouped into:
\begin{center}
\begin{varwidth}{\textwidth}
\item \emph{list} $\rightarrow$ \emph{list} \Plus\ \emph{digit} | \emph{list} \Minus\ \emph{digit} | \emph{digit}
\end{varwidth}
\end{center}
According to conventions, the terminals of the grammar are symbols:
\begin{center}
\begin{varwidth}{\textwidth}
\item \Plus\ \Minus\ 0 1 2 3 4 5 6 7 8 9
\end{varwidth}
\end{center}

The nonterminals are the italicized names \emph{list} and \emph{digit}, with list being the start symbol because its production are given first.

We say a production is for a nonterminal if the nonterminal is the head of the production. A string of terminals is a sequence of zero or more terminals. The string of zero terminals, written as $\epsilon$, is called the empty string.










\end{document}

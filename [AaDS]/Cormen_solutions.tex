\documentclass[12pt,a4paper,oneside,draft]{report}
\usepackage{ragged2e}
\usepackage{titlesec}
\usepackage{hyperref}

\newcommand{\fakesection}[1]{% instead of /section
  \par\refstepcounter{section}% Increase section counter
  \sectionmark{#1}% Add section mark (header)
  \addcontentsline{toc}{section}{\protect\numberline{\thesection}#1}% Add section to ToC
  % Add more content here, if needed.
}

\overfullrule=0pt

\begin{document}

\titleformat{\chapter}[display] % Removing chapter header
{\normalfont\bfseries}{}{0pt}{\Large}

\begin{center}
	Solutions for Introduction to Algorithms 2022, 4th edition. \\
	\href{https://ocw.mit.edu/courses/6-006-introduction-to-algorithms-fall-2011/pages/syllabus/}{link to MIT open course}
\end{center}

\chapter{}
\fakesection{}
\subsection{Describe your own real-world example that requires sorting. Describe one that
requires finding the shortest distance between two points.}
Sorting product tables by various parameters: date, price, etc. AABB collisions needs 
to know shortest distance between two points in rectangle-circle case.

\subsection{Other than speed, what other measures of efficiency might you need to consider in
a real-world setting?}
Memory.

\subsection{Select a data structure that you have seen, and discuss its strengths and limitations.}
Hash table. Operations \texttt{Search}, \texttt{Delete} and \texttt{Insert} 
take $\mathcal{O}(1)$ time on average. However, database can degrade if it goes through a large number of collisions.

\newpage

\subsection{How are the shortest-path and traveling-salesperson problems given above similar?
How are they different?}
Both problems regard path finding with minimum cost. \\
The are different in few things:
\begin{itemize}
	\item TSP is NP-complete while SPP is known P-complex
	\item UPSOLVE
\end{itemize}





\end{document}
